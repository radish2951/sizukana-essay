% ---------------------------------------------------------------
% このファイルは、本テンプレート用のメインTeXソースです。
% 書籍全体の構成や本文、タイトルページ、目次、各種設定を記述しています。
% このファイルを書き換えることで、オリジナルの本を作成できます。
% ---------------------------------------------------------------

\documentclass[
  book, tate,           % 縦書きレイアウト、見開きページ設定
  paper=a6,             % 用紙サイズをA6に指定
  fontsize=13Q,         % 欧文フォントサイズを13Qに設定
  jafontsize=13Q,       % 和文フォントサイズを13Qに設定
  line_length=39zw,     % 1行あたり全角39文字分の長さ
  number_of_lines=16,   % 1ページあたり16行
  gutter=13mm,          % ノド側(綴じ側)の余白を13mmに設定
  foot_space=10mm,      % 地(下部)余白を10mmに設定
  baselineskip=1.6zw,   % 行送りを全角1.6文字分に設定
  headfoot_verticalposition=1zw, % ノンブル(ページ番号)と本文の間隔を全角1文字分に設定
  hanging_punctuation   % 句読点のぶら下げを有効化
]{jlreq}

% スタイルを読み込み
\usepackage{essaystyle}

% 書籍情報
\newcommand{\booktitle}{透明な私と桃色の君}      % 書籍タイトル
\newcommand{\booksubtitle}{White Peach Prismatica} % サブタイトル
\newcommand{\authorname}{池田大輝}                % 著者名

\begin{document}

% タイトルページの作成
\begin{titlepage}
  \begin{minipage}<y>{\textheight}
    \centering
    \vspace*{0.5\textheight} % ページ中央にタイトル等を配置
    {\LARGE \booktitle}\\[1em] % 書籍タイトルを大きく表示
    {\large \booksubtitle}\\[2em] % サブタイトルをやや小さく表示
    {\normalsize \authorname} % 著者名を通常サイズで表示
  \end{minipage}
\end{titlepage}

\chapter*{このテンプレートについて} % *付きなので目次から除外される

\LaTeX{}というものをご存知だろうか。あるいは\TeX{}でも良い。
この謎の文字はなんと読めば良いのか。ChatGPTに聞いてみよう。

\

\begin{quote}
それは「テフ」って読むんですよ。元々は「技術」を意味するギリシャ語に由来しています。
ちなみに\LaTeX{}は「ラテフ」または「ラテック」なんて呼ばれています。
\end{quote}

\

だそうだ。妙に馴れ馴れしいのが気になるけれど、たぶん合っていると思う。

\LaTeX{}は、このような文章を美しく作成することのできるソフトウェアである。
このような文章とは、いま、あなたが読んでいる、まさにこの文章のことである。
\LaTeX{}は、通常は科学的な論文やレポートの執筆に使われる。たとえば、

\[
  \frac{1}{\pi} = \frac{2\sqrt{2}}{9801}
  \sum_{n=0}^{\infty}
    \frac{(4n)! \, (1103 + 26390 n)}{(n!)^{4}\, 396^{4n}}
\]

のような感じで数式を書くことができる。ただし、縦書きの中に横に書かれているので読みづらい。

\

そう、これは日本語文章を縦書きするために用意された\LaTeX{}テンプレートである。
すなわち、このテンプレートをダウンロードして、中身の文章さえ書き換えてしまえば本が出来上がる、
という代物である。
正確に言えば、中身を書き換えるだけではだめで、
書き換えたこのファイル(\texttt{main.tex})をコンパイルする必要がある。
LuaLaTeXと呼ばれる\LaTeX{}の一種を使用してこのファイルをPDFに変換することができる。
細かい手順はここでは触れない。調べれば情報は出てくるので、各自、調べていただきたい。

\

インストールや環境構築がやや手間であるけれど、一度環境を整えてしまえば、
あとは同じフォーマットで本を量産できるのが\LaTeX{}の良いところである。
細かいデザインやレイアウトを気にせずにただ文章を書いていたい、
というのは多くの字書きが共感するところだろう。

そんなことはない? デザインにもこだわりたい? 大丈夫。
\LaTeX{}はあらゆるニーズにも応え得る驚くべき柔軟性を兼ね備えている。
が、ここに書くには余白が狭すぎた。

\cleardoublepage % 本文開始前に改ページ

\tableofcontents % 目次を追加

\cleardoublepage % 本文開始前に改ページ

\chapter{第一章 冒険のはじまり}

チュートリアルは終わり。あとは、書くだけだ。

\

さあ、書こう。書いて書いて、書きまくるのだ。

% 本の最後にある奥付(著者情報などが書かれたページ)を別ファイルから読み込むことができる。
% 今更だけれど、%から始まるテキストはコメント、すなわち本には現れない箇所として扱われる。
% では、%を含むテキストはどう書けば良いのか? \LaTeX{}をめぐる冒険が、はじまる——。
% ---------------------------------------------------------------
% このファイルは書籍の奥付(著者情報や発行日、連絡先、著作権表示など)を
% 記述するためのものです。内容やレイアウトは自由に編集できます。
% ---------------------------------------------------------------

% 改ページ
\cleardoublepage{}

% ヘッダーとフッターを奥付に表示しない
\pagestyle{empty}

\begin{minipage}<y>[b][\textwidth - 10mm]{\textheight} % 幅をページの高さいっぱいに設定

\fontsize{7pt}{10pt}\selectfont % フォントサイズを適用

% 著者情報などを自由に書いてください。
% ここに書かれているのはこのテンプレートの作者・池田の情報ですので、くれぐれもこのまま使わないようご注意ください。
% 必ず著者ご本人の情報をご記入ください。
著者|\textbf{池田大輝(Daiki Ikeda)}。ゲーム作家・シナリオライター。1993年、秋田県秋田市生まれ。
東京工業大学(現東京科学大学)生命理工学部卒。
高校時代に映画同好会を立ち上げ、映画制作を開始。
3人で制作したストップモーションアニメ『こくせん 黒板戦争』のヒットをきっかけに、
『岩井俊二のMOVIEラボ(Eテレ)』をはじめとする数多くのテレビ番組に出演。
大学卒業後は日本ファルコム株式会社に入社。
のちにギャルゲーというジャンルの存在を知り、ゲーム制作を始める。
主な作品:『さくらいろテトラプリズム』(主演:吉岡茉祐、桜咲千依、山岡ゆり、和久井優)、
『C.V. 私と私の中の人 The Fourth Insider』(主演:宮白桃子)。

\vfill % 可能な限りの縦方向の空白を入れて奥付をページ下部へ

% 奥付情報をお好みでカスタマイズしてください。
{\Large \booktitle}\\[1em]
{\large \booksubtitle}\\[1em]
{\normalsize \authorname}\\
\copyright{} Daiki Ikeda 2025\\[1em]
2025年5月11日 第1刷発行\\[1em]
メールアドレス:\url{daiki@daiki.pink}\\
X (Twitter):\url{@radish2951} \url{@bloomingspectr}\\[1em]

\fontsize{6pt}{9pt}\selectfont % フォントサイズを適用

% こちらの文言もご自由に変更してください。
本書の内容の一部または全部を、著作権者の許可なく複製、転載、配布、改変、または販売を行うことは、
著作権法により固く禁じられています。これらの行為は法律に基づき罰せられる場合がありますので、十分ご注意ください。

\end{minipage}


\end{document}
